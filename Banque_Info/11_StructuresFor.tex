\element{StructuresFor}{
\begin{question}{for01}  % genumsi nreveret
On souhaite écrire un programme affichant tous les entiers multiples de 3 entre 6 et 288 inclus.
Quel code est correct ?
	\begin{multicols}{2}	
	\begin{reponses}	
	\bonne{$\;$\lstinputlisting{For_01_d.py}}
	\mauvaise{$\;$\lstinputlisting{For_01_a.py}}
	\mauvaise{$\;$\lstinputlisting{For_01_b.py}}
	\mauvaise{$\;$\lstinputlisting{For_01_c.py}}
	\end{reponses}
	\end{multicols}
\end{question}\\}

\element{StructuresFor}{
\begin{question}{for02}  % genumsi nreveret
	On a saisi le code suivant :
	\lstinputlisting{For_02.py}
	Quelle est la valeur de \lstinline{a} après l’exécution du code ?
\begin{multicols}{4}	
	\begin{reponses}	
	\bonne{26}
	\mauvaise{18}
	\mauvaise{18.0}
	\mauvaise{26.0}
	\end{reponses}
	\end{multicols}
\end{question}\\}

\element{StructuresFor}{
\begin{question}{for03}% osupk8
\texttt{Pour i allant de 0 à 9} s'écrit :	
\begin{multicols}{2}
	\begin{reponses}	
	\bonne{\lstinline{for i in range(10) :}}
	\mauvaise{\lstinline{for i in range(8) :}}
	\mauvaise{\lstinline{for i in range(9) :}}
	\mauvaise{\lstinline{for i in range(11) :}}
	\end{reponses}
	\end{multicols}
\end{question}\\}

\element{StructuresFor}{
\begin{question}{for04}% osupk8
\texttt{pour i allant de 1 à 10} s'écrit :
\begin{multicols}{2}
	\begin{reponses}	
	\bonne{\lstinline{for i in range(1,11) :}}
	\mauvaise{\lstinline{for i in range(10) :}}
	\mauvaise{\lstinline{for i in range(1,10) :}}
	\mauvaise{\lstinline{for i in range(0,10) :}}
	\end{reponses}
	\end{multicols}
\end{question}\\}

\element{StructuresFor}{
\begin{question}{for05}% planchet.d
On a saisi le code suivant : 
	\lstinputlisting{For_05.py}
Qu'affiche le programme python ?
	\begin{reponses}	
	\bonne{4.}
	\mauvaise{5.}
	\mauvaise{0 puis 1 puis 2 puis 3 puis 4.}
	\mauvaise{0 puis 1 puis 2 puis 3 puis 4 puis 5.}
	\end{reponses}
\end{question}\\}

\element{StructuresFor}{
\begin{question}{for06}% sgenre
Qu'affiche le script suivant : 
	\lstinputlisting{For_06.py}
\begin{multicols}{4}
	\begin{reponses}	
	\bonne{15}
	\mauvaise{6}
	\mauvaise{20}
	\mauvaise{11}
	\end{reponses}
	\end{multicols}
\end{question}\\}

\element{StructuresFor}{
\begin{question}{for07}% 
\texttt{for i in range(5) :} signifie que \texttt{i} prend les valeurs :
	\begin{reponses}	
	\bonne{0, 1, 2, 3, 4.}
	\mauvaise{1, 2, 3, 4, 5.}
	\mauvaise{5, 4, 3, 2, 1.}
	\mauvaise{4, 3, 2, 1, 0.}
	\end{reponses}
\end{question}\\}

\element{StructuresFor}{
\begin{question}{for08}%
Quelles sont les valeurs prises successivement par la variable \texttt{i} dans la boucle for ci-dessous ?
	\lstinputlisting{For_08.py} 
\begin{multicols}{4}
	\begin{reponses}	
	\bonne{0, 1, 2.}
	\mauvaise{0, 1, 2, 3.}
	\mauvaise{1, 2, 3.}
	\mauvaise{1, 2, 3, 4.}
	\end{reponses}
	\end{multicols}
\end{question}\\}


%\element{StructuresFor}{
%\begin{question}{for}% 
%\begin{multicols}{4}
%	\begin{reponses}	
%	\bonne{}
%	\mauvaise{}
%	\mauvaise{}
%	\mauvaise{}
%	\end{reponses}
%	\end{multicols}
%\end{question}\\}