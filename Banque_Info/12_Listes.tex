\element{Listes}{
\begin{question}{list01}  % genumsi nreveret
	Quelle est le résultat de :  \lstinline{[ (a,b) for a in range(3) for b in range(a) ]} ?
%\begin{multicols}{4}	
	\begin{reponses}	
	\bonne{\lstinline{[(1,0),(2,0),(2,1)]}}
	\mauvaise{\lstinline{[(1,0),(2,1),(2,1)]}}
	\mauvaise{\lstinline{[(1,0),(2,1),(3,2)]}}
	\mauvaise{\lstinline{[(0,0),(1,1),(2,2)]}}
	\end{reponses}
%\end{multicols}
\end{question}\\}

\element{Listes}{

\begin{question}{list02}  % genumsi osupk8
Soit la liste :  \lstinline{liste_pays = ['France','Allemagne','Italie','Belgique','Pays Bas']}. Que renvoie l'instruction : \lstinline{'Belgique' in liste_pays}.	
	\begin{reponses}	
	\bonne{\lstinline{True}}
	\mauvaise{\lstinline{False}}
	\mauvaise{\lstinline{liste_pays}}
	\mauvaise{\lstinline{['France','Allemagne','Italie','Belgique','Pays Bas']}}
	\end{reponses}
\end{question}\\}


\element{Listes}{
\begin{question}{list02}  % genumsi osupk8
	Soit la liste :  \lstinline{liste_pays = ['France','Allemagne','Italie','Belgique','Pays Bas']}. Que renvoie l'instruction : \lstinline{liste_pays[2]}.	
	\begin{reponses}	
	\bonne{Italie}
	\mauvaise{France}
	\mauvaise{Allemagne}
	\mauvaise{Belgique}
	\end{reponses}
\end{question}\\}
%
%
%\element{Listes}{
%\begin{question}{list01}  % genumsi nreveret
%\begin{multicols}{4}	
%	\begin{reponses}	
%	\bonne{}
%	\mauvaise{}
%	\mauvaise{}
%	\mauvaise{}
%	\end{reponses}
%	\end{multicols}
%\end{question}\\}